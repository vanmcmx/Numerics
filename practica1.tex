\documentclass[addpoints,12pt]{exam}
\usepackage{mathptmx,amsmath,amsthm,amssymb,mathtools,bm}
\usepackage[utf8]{inputenc}    % escribir con acentos
\usepackage[spanish,mexico]{babel}
\usepackage{verbatim}
\usepackage{tikz}
\usepackage{hyperref}

\newcommand*\circled[1]{
    \tikz[baseline=(char.base)]{
        \node[shape=circle,draw,inner sep=0pt] (char) {#1\strut}
    }\kern-3pt
}

\renewcommand{\thequestion}{\fbox{\Large\bf\arabic{question}}}
\renewcommand{\thepartno}{\large\bf\alph{partno}}
\renewcommand{\partlabel}{\circled{\thepartno}}
\renewcommand{\questionlabel}{\thequestion}
\bracketedpoints
\pointpoints{pt}{pts}


\title{\vspace{-2.0cm}
{\Large \textbf{$\bm 1^\circ$ Práctica de Análisis Numérico}} \\
 \vspace{-0.4cm} }

\author{Profesora: Ursula Iturrarrán\vspace{-0.5cm}}
\date{Semestre 2018-2 \vspace{-0.5cm}} 

\pagestyle{head}


\begin{document}

\maketitle
\hrule

\begin{questions}

\fullwidth{\Large \textbf{Parte  Téorica de Aritmética de Punto Flotante}}
\vspace{0.2cm}

\question[1] 
Considera el sistema de punto flotante $F(\beta=10,p=7,L=-13,U=11)$
\begin{parts}
\part Incluyendo al cero, ¿cuál es la cardinalidad de este sistema?
\part Hallar el número normalizado positivo más grande (el overflow) y  el 
número normalizado positivo más pequeño (el underflow)
\part Si usamos redondeo al más cercano, ¿Cuál es el épsilon de la máquina $\epsilon_{\text{mach}}$?
\part Use redondeo al más cercano para hallar 
$$\text{fl}\left(\text{fl}\left(\dfrac{1}{3}\right)+
  \text{fl}\left(\dfrac{1}{5}\right)\right)
  \quad\text{y}\quad
  \text{fl}\left(\text{fl}\left(\dfrac{1}{3}\right) -
  \text{fl}\left(\dfrac{1}{5}\right)\right)$$
 en este sistema de punto flotante 
\end{parts}

\question[1] 
Considera el sistema de punto flotante $F(\beta=2,p=8,L=-10,U=10)$
\begin{parts}
\part ¿cuántos números tienen exponente igual a 10?
\part Hallar el overflow y el underflow. Convierta los números obtenidos a sistema
 decimal.
\part Si usamos truncamiento, ¿Cuál es el épsilon de la máquina
 $\epsilon_{\text{mach}}$?
\part  Si usamos truncamiento, ¿cúal es el valor de 
$float\left((101010.0011111)_2\right)$?
\part Convierta 67.13 a binario y calcule $float(67.13)$ usando truncamiento
\part Si usamos truncamiento, ¿cúal es el valor de 
 $(1111.00011)_2\oplus (10001.1001)_2$ en este sistema?
\end{parts}


\question[1] 
Formatos Simple y Doble del IEEE
\begin{parts}
\part Convierta la precisión $\rho$ del formato doble IEEE (64 bits) a binario
\part Convierta su año de nacimiento al formato simple IEEE (32 bits).  Use
representación sesgada para el exponente y truncamiento si es necesario.
\part Hallar el número flotante más grande del formato simple IEEE  
con exponente igual a $(110011)_2$ y conviertelo a sistema decimal.
\end{parts}

\question[1] 
 Errores de Cancelación
\begin{parts}  
\part Dar una fórmula alternativa a  $\dfrac{1-\sqrt5}{2}$ que evite errores 
de cancelación.
\part Dar una fórmula que aproxime $\exp(-9)$ usando serie de Taylor hasta el 
$5^\circ$ término. Evita errores de cancelación.
\part Dar una fórmula para calcular la varianza muestral de $N+1$ observaciones 
$x_0,\hdots,x_N>0$ que solamente realize una cancelación.
\end{parts}

\fullwidth{\Large \textbf{Parte Téorica de Álgebra Lineal Numérica}}

\question[2]
 Normas vectoriales y Normas matriciales
\begin{parts}  
\part Encuentre la figuras que forman los siguientes conjuntos

$B_1=\left\{
\begin{pmatrix} x \\ y \end{pmatrix}\in\mathbb R^2:
\left\|\begin{pmatrix} x \\ y \end{pmatrix}\right\|_1\leq 1\right\}$

$\overline{B_1}=\left\{
\begin{pmatrix} x \\ y \end{pmatrix}\in\mathbb R^2:
\left\|\begin{pmatrix} x \\ y \end{pmatrix}\right\|_1=1\right\}$

$B_{\infty}=\left\{
\begin{pmatrix} x \\ y \end{pmatrix}\in\mathbb R^2:
\left\|\begin{pmatrix} x \\ y \end{pmatrix}\right\|_{\infty}\leq 1\right\}$

$\overline{B_{\infty}}=\left\{
\begin{pmatrix} x \\ y \end{pmatrix}\in\mathbb R^2:
\left\|\begin{pmatrix} x \\ y \end{pmatrix}\right\|_{\infty}= 1\right\}$

$C =\left\{
\begin{pmatrix} x \\ y \end{pmatrix}\in\mathbb R^2:
\left\|\begin{pmatrix} x \\ y \end{pmatrix}\right\|_2=1\right\}$

\part Sea 
$$A = 
\left(\begin{array}{rr}
    -7 & 0 \\ 0 & 1 
  \end{array}\right)$$
Encuentre la figuras que forman los siguientes conjuntos

$S_1=\left\{
\begin{pmatrix} x \\ y \end{pmatrix}\in\overline{B_1}:
\left\|A\begin{pmatrix} x \\ y \end{pmatrix}\right\|_1=1\right\}$

$S_{\infty}=\left\{
\begin{pmatrix} x \\ y \end{pmatrix}\in\overline{B_{\infty}}:
\left\|A\begin{pmatrix} x \\ y \end{pmatrix}\right\|_{\infty}=1\right\}$

$S_2 =\left\{
\begin{pmatrix} x \\ y \end{pmatrix}\in C:
\left\|A\begin{pmatrix} x \\ y \end{pmatrix}\right\|_2=1\right\}$

Use estas figuras para hallar $\|A\|_1,\|A\|_{\infty}$ y $\|A\|_2$, 
respectivamente
\end{parts}
En ambos incisos explica tus pasos y dibuja las figuras usando el software
{\tt Geogebra}

\newpage

\question[2]
 Normas matriciales y número de condición
\begin{parts}  
\part
Sea $A$ una matriz real de tamaño $m\times n$.
Pruebe que $$\|A\|_1=\max_{1\leq j\leq n}\sum_{i=1}^m|a_{i,j}|\quad\text{y}\quad 
\|A\|_{\infty}= \max_{1\leq i\leq m}\sum_{j=1}^n|a_{i,j}|$$
\part Calcule el número de condición de
 $$L=
 \begin{bmatrix}
  1 & 0 & 0 \\ 1 & 1 & 0 \\ 1 & 1 & 1
 \end{bmatrix}$$
en normas 1 e $\infty$ usando las fórmulas del inciso (a). 

¿la inversa de $L$ es una matriz triangular superior o inferior?
\part Calcule el número de condición de
 $$T = 
  \left(\begin{array}{cr}
    -2^{-5} & 2^5 \\ 0 & -2^5 
  \end{array}\right)$$
en normas 1 e $\infty$ usando las fórmulas del inciso (a).

¿la inversa de $T$ es una matriz triangular superior o inferior?
\end{parts}

\question[2]
Considera
$$A=
\left[\begin{array}{rrr}
  1 & -1 & 0   \\  -1 & 2 & -1  \\ 0  &-1  &2  
\end{array} \right]\quad\text{y}\quad
\bm b = \left[\begin{array}{r} 0 \\ 1 \\ -1 \end{array}\right].$$

\begin{parts}  
\part Investiga al menos 2 maneras distintas de saber cuando una matriz simétrica
es positiva definida.
\part Prueba que $A$ es positiva definida ya sea por definición o usando inciso (a)
\part Calcula la Factorización de Cholesky de $A$ y usala para resolver 
$A\bm x=\bm b$ 
\part Calcula la Factorización LU de $A$ con pivoteo por renglones  y usala para 
resolver $A\bm x=\bm b$ 
\part Calcula la Factorización $LDL^T$ de $A$
\end{parts} 

\end{questions}
\newpage

\underline{Indicaciones Generales:}
\begin{itemize}
\item {\bf Usa \LaTeX para elaborar un archivo PDF con las respuestas de 
 esta práctica}
\item Puedes registrarte en
\url{https://www.overleaf.com} o en \url{https://latexbase.com} para usar 
\LaTeX en linea sin necesidad de instarlar software ni paquetes adicionales. 
\item Puedes usar {\tt Geogebra} en linea en 
 \url{https://www.geogebra.org/geometry}. Guarda tus archivos como {\tt .gbb} 
 y {\tt .pdf}
\item Poner los pasos principales de tus calcúlos y conversiones, así como
explicaciones en el PDF que vas a entregar.
\item Tutorial de \LaTeX:

\url{https://www.overleaf.com/latex/learn/free-online-introduction-to-latex-part-1}
\end{itemize}
\end{document}
